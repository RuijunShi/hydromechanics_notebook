\documentclass[fontset=windows]{report}
\usepackage[T1]{fontenc}
\usepackage[utf8]{inputenc}
\usepackage[margin=1in]{geometry}%设置边距,符合Word设定
\usepackage{ctex}
\usepackage{setspace}
\usepackage{subfiles}
\usepackage{lipsum}
\usepackage{amsmath,amssymb}
\usepackage{txfonts}
\usepackage{array}
\usepackage{lmodern}
\usepackage{iftex}
\usepackage{textcomp}
\usepackage{graphicx}%插入图片
\usepackage{lmodern}
\usepackage{xcolor}
\usepackage{textcomp} % provide euro and other symbols
%\graphicspath{{Figures/}}%文章所用图片在当前目录下的 Figures目录
\usepackage{hyperref} % 对目录生成链接,注:该宏包可能与其他宏包冲突,故放在所有引用的宏包之后
\usepackage{parskip}
\usepackage{xurl}
\usepackage{upquote}
\hypersetup{colorlinks = true,  % 将链接文字带颜色
	bookmarksopen = true, % 展开书签
	bookmarksnumbered = true, % 书签带章节编号
	pdftitle = 统计学笔记, % 标题
	pdfauthor =RuijunShi} % 作者
\bibliographystyle{acm}% 参考文献引用格式
\newcommand{\upcite}[1]{\textsuperscript{\cite{#1}}}
\renewcommand{\contentsname}{\centerline{Contents}} %经过设置word格式后,将目录标题居中

\title{\heiti\zihao{1} 流体力学笔记}
\author{\songti\zihao{3} Ruijun Shi\thanks{GitHub : \href{https://github.com/RuijunShi}{https://github.com/RuijunShi}}}
\date{\songti\zihao{3} \today}

\begin{document}
  \maketitle
  \thispagestyle{empty}
  \tableofcontents


\chapter{Introduction}

\section{描述流体}

密度\(\rho\), 压力\(p\), 温度\(T\), 速度\(\vec{u}\)

\section{稳态和非稳态流体}

\[\frac{\partial}{\partial t}=0\]

\section{二维流动}

two components ( plane flow )

two coordinates ( e.g. : \(\partial / \partial \phi = 0\) )

\section{流线}

At a \textbf{fixed} time

\[\frac{dx}{u_x}=\frac{dy}{u_y}=\frac{dz}{u_z}=const\]


\section{物质导数和迁移 material derivative and advection}
\begin{equation}
  \begin{aligned}
    \frac{\mathrm{d}f}{\mathrm{d}t}
    &=\frac{\partial f}{\partial t}+
    \frac{\partial f}{\partial x}\frac{\partial x}{\partial t}+
    \frac{\partial f}{\partial y}\frac{\partial y}{\partial t}+
    \frac{\partial f}{\partial z}\frac{\partial z}{\partial t}\\
    &=  \frac{\partial f}{\partial t}+
    u_x \frac{\partial f}{\partial x}+
    u_y \frac{\partial f}{\partial y}+
    u_z \frac{\partial f}{\partial z}\\
    &=\frac{\partial f}{\partial t}+(\vec{u}\cdot\vec{\nabla})f
    \end{aligned}
\end{equation}


其中:

\(\frac{df}{dt}\) is material derivative

\(\frac{\partial f}{\partial t}\) is local rate

\((\vec{u}\cdot\vec{\nabla})f\) is advection rate

\section{拉格朗日描述和欧拉描述}
If f is \(\vec{u}\) :

\[\frac{\mathrm{d}\vec{u}}{\mathrm{d}t}=\frac{\partial \vec{u}}{\partial t}+(\vec{u}\cdot\vec{\nabla})\vec{u}\]

左边为拉格朗日描述,右边为欧拉描述


\section{不可压缩流体(incompressible fluid)}

\[\oint\vec{u}\cdot\vec n \ ds=0\]

微分形式

\[\vec\nabla \cdot \vec u=0\]

\section{mass conservation 质量守恒}

对于一个流体微团, 其质量变化率为:

\[\frac{\partial}{\partial t}\int\rho dV
=-\oint\rho\  \vec u\cdot\vec n\ ds=-\int\nabla\cdot(\rho\vec u)dV\]

右边为流体微团质量的变化率; 右边为流体微团的面积微元的法向上,
物质流出/流入速率, 变成微分形式:

\[\frac{\partial \rho}{\partial t}+\vec \nabla\cdot(\rho\vec u)=0\]

等式左边第二项展开:
\begin{equation}
  \begin{aligned}
    \frac{\partial \rho}{\partial t}
    +(\vec u \cdot \vec \nabla)\rho+\rho(\vec \nabla\cdot\vec u)&=0\\
    \frac{d \rho}{dt}+\rho(\vec \nabla\cdot\vec u)&=0
    \end{aligned}
\end{equation}


换个角度看质量守恒( 连续方程 ), 在拉格朗日视角,
一个流体微元的体积为\(\delta \tau=\delta x\delta y\delta z\),
质量为\(\delta m = \rho \delta \tau\); 在运动过程中,
该流体微元质量总是守恒的.

\[\frac{\rm d}{{\rm d}t}(\delta m)=\frac{\rm d}{{\rm d}t}(\rho\delta \tau)=0\]

展开, 得到:

\[\frac 1 \rho \frac{{\rm d}\rho}{{\rm d}t}+\frac 1 {\delta \tau}\frac{\rm d}{{\rm d}t}(\delta \tau)=0\]

其中, \(\frac 1 {\delta \tau}\frac{\rm d}{{\rm d}t}(\delta \tau)\)
为流体膨胀速率与原来体力比, 表征流体的体膨胀速度., 写开变为:

\[\frac 1 {\delta \tau}\frac{\rm d}{{\rm d}t}(\delta \tau)=\nabla\cdot u\]

整理得到连续方程. 对于不可压缩流体:
\begin{equation}
  \begin{aligned}
    \rho(\vec \nabla\cdot\vec u)&=0\\
    \frac{d \rho}{dt}&=0
    \end{aligned}
\end{equation}



\section{Vorticity 涡度/旋度}

\[\vec \omega = \vec \nabla\times\vec u\]

由斯托克斯公式\(\iint_s\nabla\times f\cdot n {\rm d}s=\oint_l f \cdot dl\)对速度环量和涡度积分:

\[\oint\vec ud\vec l = \int \vec \nabla\times \vec u\ d\vec s=\int \vec\omega\  d\vec s\\\]

\(\Omega\)为流体运动的角速度,积分得到:

\[\Omega R\cdot 2\pi R = \omega\cdot \pi R^2\]

即:

\[\omega=2\Omega\]

涡度是局地旋转角速度的两倍。


\section{potential flow 势场}

当\(\vec \omega=0\):

\[\vec u = \vec \nabla\psi\]

\section{算符:梯度,散度和旋度}



\chapter{Inviscid fluid}
\section{Euler equation}

momentum equation:
\begin{equation}
  \begin{aligned}
    \frac{\partial}{\partial t}\int \rho u_i dV
    &=-\oint\rho u_i(\boldsymbol u \cdot\boldsymbol n)ds
    -\oint pn_id\boldsymbol s +\int \rho f_idV\\
    &=-\int \boldsymbol \nabla (\rho u_i\boldsymbol u)dV-
    \int \frac{\partial p}{\partial x_i}dV+\int\rho f_i dV
    \end{aligned}
\end{equation}


这个式子运用了雷诺输运方程.
左边的式子是一个流体微团在\(u_i\)方向上的动量随着时间的变化率。计算可知左边第一项为动量的量纲。右边第一项为流体的流出率,表征流体微团因为流进流出导致的动量变化。\((u\cdot n)ds\)表示在一个很小的面元ds上的流体速度u在其法向上的大小,表征流体微团在方向n的扩散速度,可以表征流体的动量守恒.

有:

\[\frac{\partial}{\partial t}(\rho u_i)+\boldsymbol\nabla (\rho u_i \boldsymbol u)=-\frac{\partial p}{\partial x_i}+\rho f_i\]

其中:

\begin{align*}
LHS
&=\rho\frac{\partial u_i}{\partial t}+u_i\frac{\partial \rho}{\partial t}+\rho(\boldsymbol{u\cdot\nabla})u_i+u_i \boldsymbol\nabla(\rho \boldsymbol u)\\
&=\rho\left(\frac{\partial u_i}{\partial t}+(\boldsymbol{u\cdot\nabla})u_i\right)\\
&=\rho\frac{du_i}{dt}
\end{align*}

对所有分量:

\[\frac{d\boldsymbol u}{dt}=-\frac{1}{\rho}\boldsymbol\nabla p+\boldsymbol f\]

\section{Close the equation}

流体方程的参数:\(\rho\ \ \vec u \ \ p\) 一共五个参数。

连续方程:

\[\frac{d \rho}{dt}+\rho(\vec \nabla\cdot\vec u)=0\]

欧拉方程:

\[\frac{d\boldsymbol u}{dt}=-\frac{1}{\rho}\boldsymbol\nabla p+\boldsymbol f\]

若\(\rho=\) const,则上述方程封闭。若该流体是可压缩的,引入状态方程:

\[p=\rho R T\]

\section{Bernoulli theorem}

steady flow : \(\partial u/\partial t=0, \rho=\) const.

\begin{align}
&\vec u \cdot \vec \nabla u =- \frac{1}{\rho}\vec \nabla p+\vec g \\
&\vec u\cdot \vec \nabla u = \vec \nabla(u^2/2)-\vec u\times\vec \omega\\
&-\nabla(p/\rho+u^2/2+gz)+u\times\omega = 0
\end{align}

记\(H =p/\rho+u^2/2+gz\) :

\[-\nabla H+u\times \omega=0\]

两边同时乘\(u\) :

\[u\cdot\nabla H=0\\
\partial H/\partial s = 0\]

因此

\[H = \frac{u^2}{2}+\frac{p}{\rho}+gz=const\]

\section{Vorticity equation}

\[\frac{\partial u}{\partial t}=-\nabla H+u\times\omega\]

\[\frac{\partial \omega}{\partial t}=\nabla\times(u\times\omega)=(\omega\cdot \nabla)u-(u\cdot\nabla)\omega\\
\frac{d\omega}{dt}=(\omega\cdot\nabla)u\]

涡度形变:

\[(\omega\cdot\nabla)u \simeq \omega_z\frac{\partial u_z}{\partial z}\]

\section{Kelvin theorems}

\[\Gamma=\oint u\cdot dl=\int\omega\cdot ds\]

\[\frac{d\Gamma}{dt}=\frac{d}{dt}\int_{s(t)}\omega\cdot ds=\int_{s(t)} \left[\frac{\partial \omega}{\partial t}-\nabla\times(u\times\omega)\right]ds=0\]

\[\frac{d}{dt}(d\vec l)=\vec u(\vec x+d\vec l)-\vec u(\vec x)=(d\vec l\cdot \nabla)\vec u(\vec x)\]

\[\frac{d\omega}{dt}=(\omega\cdot\nabla)u\]

如果\(\omega\)方向与\(l\)方向平行, 那之后将会一直平行. 开尔文定理:

\[\frac{d\Gamma}{dt}=0 \label{kel}\]

也就是说涡度强度不随时间变化, 或者说无外力下无旋流体不可能获得涡度.



\chapter{Viscous fluid}

\section{Velocity decompostion}

\[d\vec u = (d\vec r\cdot\vec \nabla)\vec u\]

\begin{align}
du_i &= dx\frac{\partial u_i}{\partial x}+dy\frac{\partial u_i}{\partial y}+dz\frac{\partial u_i}{\partial z}\\
&=\sum_{j=1}^{3}\frac{\partial u_i}{\partial x_j}dx_j\\
&=\frac{\partial u_i}{\partial x_j}dx_j
\end{align}

Einstein summation 爱因斯坦求和约定 : 相同指标求和

一个张量可以写成对称部分和反称部分

\begin{align}
du_i &=\frac{\partial u_i}{\partial x_j}dx_j\\
&=\frac{1}{2}
\left(\frac{\partial u_i}{\partial x_j}
+\frac{\partial u_j}{\partial x_i}\right)dx_j+
\frac{1}{2}
\left(\frac{\partial u_i}{\partial x_j}
-\frac{\partial u_j}{\partial x_i}\right)dx_j
\end{align}

其中对称部分和反称部分记为:

\[e_{ij}=\frac{1}{2}\left(\frac{\partial u_i}{\partial x_j}+\frac{\partial u_j}{\partial x_i}\right)
\label{jqxb}\]

\[\xi_{ij}=\frac{1}{2}\left(\frac{\partial u_i}{\partial x_j}-\frac{\partial u_j}{\partial x_i}\right)\]

\(e_{ij}\) : strain tensor, symmetric about i, j. (对称的)

\(\xi_{ij}\) : rotation tensor, anti-symmetric about i, j. (反称的)
我们发现这刚好和涡度对应上.

因此:

\[du_i = e_{ij}dx_j+\xi_{ij}dx_j\]

对法向形变求和:

\begin{align*}
e_{ii}
&=e_{xx}+e_{yy}+e_{zz}\\
&=\frac{1}{2}(\frac{\partial u_x}{\partial x}+\frac{\partial u_x}{\partial x})+\cdots\\
&=\vec\nabla\cdot \vec u
\end{align*}

形变张量的主对角元是速度散度。散度实际上就是面元的膨胀率. 除了法向形变,
还有剪切形变. 公式{[}\(\ref{jqxb}\){]}中当\(i\neq j\) 时为剪切形变.
表征度量流体畸变程度的物理量.

\(\xi_{ij}\) 表征旋转量,是反称的。

\[\xi_{ij}=
\frac{1}{2}
\left[
 \begin{matrix}
   0 & -\omega_z & \omega_y \\
   \omega_z & 0 & -\omega_x \\
   -\omega_y & \omega_x & 0
  \end{matrix}
  \right]\]


\section{Constitutive relation}

\(\sigma_{ij}\) : stress tensor; 从无粘流体到粘性流体:

\[\oint -p\vec n\cdot ds \longrightarrow \oint\sigma\cdot \hat n\ ds\]

而\(\sigma_{ij}\)分解为:

\[\sigma _{ij}=-p\delta_{ij}+\tau_{ij}\]

应力张量由压力和剪切组成。对于无粘流体,剪切不起作用:

\[(\sigma\cdot \hat n)_i = -p\delta_{ij}n_j=-pn_i\]

对于粘性流体:

\[\frac{{\rm d}u}{{\rm d}t}=-\frac 1 \rho \nabla p+\frac 1 \rho \nabla \tau + f\]

其中, 广义牛顿粘性假设告诉我们 :

\[\tau_{ij}=2\mu e_{ij}+\mu'(\nabla\cdot u)\delta_{ij} \\
\mu' = -\frac 2 3 \mu\]


\section{Navier-Stokes Equation}

\[\frac{{\rm d}\vec u}{{\rm d}t}=
-\frac{1}{\rho}\nabla p+
\nu\nabla^2 u+
(\nu+\nu')\nabla(\nabla\cdot u)+f\]

其中, \(\nu=\frac \mu \rho\)为运动学粘性系数, 不可压缩流体:

\[\frac{{\rm d}\vec u}{{\rm d}t}=
-\frac{1}{\rho}\nabla p+
\nu\nabla^2 u
+f \label{inNSE}\]

对于广义牛顿粘性假设的流体:

\[\frac{{\rm d}\vec u}{{\rm d}t}=
-\frac{1}{\rho}\nabla p+
\nu\nabla^2 \vec u+
\frac1 3\nu\nabla(\nabla\cdot \vec  u)+\vec f\]


\section{Boundary Condition}

\hypertarget{1-fixed-boundary}{%
\paragraph{1. fixed boundary}\label{1-fixed-boundary}}

对于固定边界, 法向速度分量为:\(\vec u\cdot\hat n\);
切向速度分量为:\(\vec u-\vec u \cdot\hat n\).

对于无粘流体:

\[\vec u\cdot\hat n=0\]

对于粘性流体,流体固定于边界时:

\[\vec u =0 \ \ and \ \     
	\begin{cases}
        \vec u\cdot \hat n=0 \\
        \vec u-\vec u\cdot \hat n=0
    \end{cases}\]

流体对边界没有应力时:
\begin{equation}
  \begin{cases}
    \vec u\cdot \hat n=0 \\
    \frac{\partial}{\partial n}(\vec u-\vec u\cdot \hat n)=0
\end{cases}
\end{equation}


\hypertarget{2-free-boundary}{%
\paragraph{2. free boundary}\label{2-free-boundary}}

运动学边界:

\[u_z=\frac{d\eta}{dt}=\frac{\partial \eta}{\partial t}+u_x\frac{\partial \eta}{\partial x}+u_y\frac{\partial \eta}{\partial y}\]

动力学边界:

\[p_{-}=p_{+}\]

即在边界上得压力相等。

\section{Normalisation}

无量纲化方程

对于无外力的不可压缩的N-S方程:

\[\frac{{\partial}\vec u}{{\partial}t}+u\cdot \nabla u=
-\frac{1}{\rho}\nabla p+
\nu\nabla^2 u \label{ncns}\]

距离尺度:\(l_0\) ; 时间尺度\(t_0\) ; 速度尺度:\(l_0/t_0\);
压力尺度:\((l_0/t_0)^2\rho\). 把尺度因子代入{[}\(\ref{ncns}\){]}:

\[\frac{l_{0} / t_{0}}{t_{0}} \frac{\partial \tilde{u}}{\partial \tilde{t}}
+\frac{\left(l_{0} / t_{0}\right)^{2}}{l_{0}} \tilde{u} \cdot \tilde{\nabla} \tilde{u} 
\\=-\frac{1}{\rho} \frac{\left(l_{0} / t_{0}\right)^{2} \rho}{l_{0}} \tilde{\nabla} \tilde{p}+ 
\quad \nu \frac{l_{0} / t_{0}}{\ell_{0}^{2}} \tilde{\nabla}^{2} \tilde{u}\]

整理得:

\[\frac{\partial \tilde{u}}{\partial \widetilde{t}}
+\tilde{u} \cdot {\nabla} \tilde{u} 
=- \tilde{\nabla} \tilde{p}+ 
\frac{\nu t_{0}}{\ell_{0}^{2}} \tilde{\nabla}^{2} \tilde{u}
\label{req}\]

其中:

\[\frac{\nu t_0}{l_0^2}=\frac{\nu}{l_0u_0}=\frac{1}{R_e}\]

\(R_e\)称为雷诺数。{[}\(\ref{req}\){]}公式则化简为:

\[\frac{\partial \tilde{u}}{\partial \widetilde{t}}
+\tilde{u} \cdot {\nabla} \tilde{u} 
=- \tilde{\nabla} \tilde{p}+ 
\frac{1}{R_e} \tilde{\nabla}^{2} \tilde{u}\]

从这个方程我们可以这样看:

\[R_e\simeq\frac{|u\cdot\nabla u|}{|\nu \nabla^2u|}\simeq\frac{u_0^2/l_0}{\nu u_0/l_0^2}=\frac{l_0u_0}{\nu}\]

雷诺数实际上表征惯性力和粘滞力之比.

\hypertarget{36-kinetic-energy-equation}{%
\section{Kinetic Energy
Equation}\label{36-kinetic-energy-equation}}

u点乘不可压缩流体的N-S方程, 并对该流体微元体积分:

\[\frac{{\rm d}}{{\rm d}t}\int\frac{u^2}{2}{\rm d}V=
-\int u \cdot(u\cdot \nabla u)dV
-\frac{1}{\rho}\int u \cdot\nabla p{\rm d}V
+\int u\cdot(\nabla \cdot \tau){\rm d}V\]

由于:
\begin{equation}
  \begin{aligned}
    \int u\cdot(u\cdot\nabla u)dV
    &=\int u_iu_j\frac{\partial u_i}{\partial x_j}{\rm d}V\\
    &=\int u_j\frac{\partial}{\partial x_j}(\frac{u^2}{2}){\rm d}V\\
    &=\int \frac{\partial}{\partial x_j}(u_ju^2/2){\rm d}V \\
    &=\oint \frac{u^2}{2}u_jn_j \rm dS\\
    &=0
    \end{aligned}
\end{equation}


因此:

\[\frac{{\rm d}}{{\rm d}t}\int\frac{u^2}{2}{\rm d}V=-2\nu\int e_{ij}e_{ij}\rm dV\]

能量耗散与粘性有关。


\section{Vorticity Equation}

对不可压缩的N-S方程{[} \(\ref{inNSE}\) {]} , 两边取速度旋度:

\[\frac{\partial\omega}{\partial t}=\nabla\times(u\times\omega)+\nu\nabla^2\omega\]

因为由粘性项,与Kelvin定理{[}\(\ref{kel}\){]}不一致,涡线重联。


\section{Microscopic Physics}

有概率密度函数\(f(r,v,t)\)

\[N=\int f {\rm d^3}r {\rm d^3}v\]

玻尔兹曼分布律:

\[\frac {\partial f}{\partial t}+v\cdot\nabla f+a\cdot\nabla_v f=C\]

其中

\[\rho(r,t)=\int mf{\rm d^3}v\]

\[\rho <v> =\int mfv{\rm d^3}v  
\label{den}\]

公式{[}\(\ref{den}\){]}对时间求偏导, 得到质量守恒.
能量守恒和动量守恒也是这个方式.



\chapter{Applications}

\section{Pressure Driven Flow}

plan posieuille flow

对于稳态流体\(du/dt=0\), 只受到压力,两边平板固定,\(p_1>p_2\)

边界条件:

\[u_x=u;\\
u_y=u_z=0\]

考虑不可压流体:

\[\nabla \cdot u =0\longrightarrow\ \ \frac{\partial u}{\partial x}=0\longrightarrow\ \  u(y,z,t)=u(y)\]

流体速度与\(x,z\)方向无关, 只与\(y\)有关. 得到:

\[-\frac{1}{\rho}\frac{\partial p}{\partial x}+\nu\frac{\partial^2 u}{\partial y^2}=0\\
-\frac{1}{\rho}\frac{\partial p}{\partial y}=0\\
-\frac{1}{\rho}\frac{\partial p}{\partial z}=0\]

得到:

\[\nu\frac{d^2u}{dy^2}=\frac{1}{\rho}\frac{dp}{dx}\]

设两平板之间的高度差为\(h\), 解得:

\[u = -\frac{h^2}{2\mu}\frac{dp}{dx}\frac{y}{h}(1-\frac{y}{h})\]

容易发现速度最大值在\(y=h/2\)上 .

\hypertarget{42-plane-couette-flow}{%
\section{Plane Couette Flow}\label{42-plane-couette-flow}}

Boundary driven flow

边界条件:

\[u(h)=U\\
u(0)=0\]

由于两边压强为0,即\(dp/dx=0\), 代入N-S方程:

\[\mu \frac{d^2u}{dy^2}=\frac{dp}{dx}\\
\mu \frac{d^2u}{dy^2}=0\]

解得:

\[u=U\frac{y}{h}\]

当两边加上压力时:

\[u=U\left[\frac{y}{h}-\frac{h^2}{2\mu}\frac{dp}{dx}\frac{y}{h}(1-\frac{y}{h})\right]\]


\section{Taylor-Couette Flow}

在一个稳态不可压缩的流体在旋转的圆柱上,边界条件为:

\[u_R=0,\ \ u_z=0,\ \ u_{\theta}=u(R),\ \ p=p(R)\]

在柱坐标上,压力梯度与离心势平衡:

\[\frac{{\rm d} p}{{\rm d}R}=\rho\frac{u^2}{R}\]

代入柱坐标下的拉普拉斯算子\(\nabla ^2=\frac{\partial^2}{\partial r^2}+\frac 1 r\frac \partial {\partial r}+\frac 1 {r^2}\frac \partial {\partial \theta ^2}+\frac \partial {\partial z ^2}\):

\[\frac{1}{R}\frac{d}{dR}(r\frac{du}{dR})-\frac{u}{R^2}=0\]

假设\(u\propto R^{\alpha}\), 当\(\alpha=\pm 1\)时,

\[u=c_1R+\frac{c_2}{R}\]

\[\Omega=c_1+\frac{c_2}{R^2}\]

对于Kepler运动:\(\Omega\propto R^{-1.5}\)


\section{Rotating Flow}

对于旋转流动,在随动坐标中展开:

\[\frac{du}{dt}\longrightarrow \frac{d u}{d t}+\Omega\times(\Omega\times r)+2\Omega\times u\]

其中右边第二项为离心项,第三项为科里奥利力项

\[\frac{\partial \vec{u}}{\partial t}+\vec{u} \cdot \vec{\nabla} \vec{u}=-\frac{1}{\rho} \vec{\nabla} p+\vec{g}-\vec{\Omega} \times(\vec{\Omega} \times \vec{r})+{2 \vec{u} \times \vec{\Omega}}\]

其中离心势写为:

\[\vec{\Omega} \times(\vec{\Omega} \times \vec{r})=\nabla(\frac{1}{2}|\vec\Omega\times \vec r|^2)\]

方程则变为:

\[\frac{\partial \vec{u}}{\partial t}+\vec{u} \cdot \vec{\nabla} \vec{u}=-\frac{1}{\rho} \vec{\nabla} p+\vec{g}-\nabla(\frac{1}{2}|\vec\Omega\times \vec r|^2)
+{2 \vec{u} \times \vec{\Omega}}\]

对比左边的第二项和右边的最后一项

\[R_0=\frac{|\vec{u} \cdot \vec{\nabla} \vec{u}|}{|2 \vec{u} \times \vec{\Omega}|}\simeq\frac{u_0^2/l_0}{2\Omega u_0}=\frac{u_0}{2\Omega l_0}\\
R_0=\frac{u_0}{2\Omega l_0}\]

也就是惯性力项/科氏力项,这个数称为Rossby数.
在地球上,假设空气是不可压缩流体, 有势函数:

\[-\frac{1}{\rho} \vec{\nabla} p+\vec{g}-\vec{\Omega} \times(\vec{\Omega} \times \vec{r})=\vec{\nabla} \psi
\label{4.4shs}\]

认为地球大气是准静态的,地球上的\(R_0<<1\),有\(du/dt=0\),
则有准地转平衡:

\[\nabla \psi+2u\times\Omega=0 \label{gb}\]

由于势函数是无旋的,我们对方程{[} \(\ref{gb}\) {]}取旋度:

\[\nabla \times (2\vec u\times\vec\Omega)=0\]

即:

\[2\vec\Omega\cdot\nabla u=0\\
\frac{\partial \vec u}{\partial l}=0\]

即速度沿着\(\vec\Omega\)方向( 旋转轴方向 )不变。


\section{Accretion Disk}

吸积盘结构 : 黑洞吸积盘, AGN, 原行星盘, 行星环, 双星系统盘的洛希极限,
X射线源

\hypertarget{1-ux4f30ux8ba1ux91cdux529bux52bfux80fd}{%
\paragraph{1.
估计重力势能}\label{1-ux4f30ux8ba1ux91cdux529bux52bfux80fd}}

\[E_p = \frac{GMm}{a}/(mc^2)=\frac{GM}{ac^2}\]

\hypertarget{2-ux8584ux76d8ux8fd1ux4f3c}{%
\paragraph{2. 薄盘近似}\label{2-ux8584ux76d8ux8fd1ux4f3c}}

在柱坐标中: \((R,\phi,z)\), 薄盘即为\(h \ll R_d \), 因此物理量为:(
\(\rho, p, T, v_R,v_\phi,v_z\)) . 其中:

\[v_z\simeq 0\\
v_{\phi}\gg v_R\\
\frac {\partial}{\partial \phi}=0\]

也就是说在盘上垂直速度很小, 且以\(\phi\)方向速度为主,
相比起来径向速度非常小. 这里引入引力势函数

\[\Psi=-GM(R^2+z^2)^{-1/2}\]

在R方向和z方向的引力为:

\[F_R=-\partial \Psi/\partial R=-GM(R^2+z^2)^{-3/2}\simeq -GMR^{-2}\]

\[F_z=-\partial \Psi/\partial z = -GMz(R^2+z^2)^{-3/2}\simeq -zGMR^{-3}\]

由于是稳态场, \(\partial/\partial t=0\), 在径向和垂直方向的运动方程为:
\begin{equation}
  \begin{array}{c}
    v_R\frac{\partial v_R}{\partial R}-\frac{v^2_\phi}{R}=-\frac 1 {\rho_g}\frac{\partial p}{\partial R}-GM/R^2\\
    0=-\frac1\rho \frac{\partial p}{\partial z}-\frac{GMz}{R^3}
    \end{array}
\end{equation}


对于理想气体状态方程:

\[p=\frac{\rho_g}{m_H}k_B T\]

密度\(\rho_g\)的分布为一个高斯分布:

\[\rho_{g} \propto \exp \left(-\frac{G M m_{H}}{2 k_{B} T R^{3}} z^{2}\right)\]

等温过程为:

\[-\frac1{\rho_g}\frac{k_B T}{m_H}\frac{d\rho_g}{dz}-\frac{GM}{R^3}z=0\]

其中:
\begin{equation}
  \begin{array}{cr}
    H_{z}=&\left(2 k_{B} T R^{3} / G M m_{H}\right)^{1 / 2} \ \ 
    \text { scale height } \\& \rho_{g} \propto \exp \left(-z^{2} / H_{z}^{2}\right)
  \end{array}
\end{equation}

通过垂直方向的运动方程, \textbf{对压力p的估计为}:

\[\frac{1}{\rho_g} \frac{p}{h} \simeq \frac{G M h}{R_{d}^{3}} \Rightarrow p \simeq \frac{G M \rho_{g} h^{2}}{R_{d}^{3}}\]

有了上面的公式, 我们对上面的公式进行量级估计:

\[v_R \ll v_\varphi;\ \ \ \ v_R\frac{\partial v_R}{\partial R}\ll v_\varphi^2/R\]

得到:

\[\left(\frac{1}{\rho _g} \frac{p}{R_{d}}\right)  /\left(\frac{G M}{R_{d}^{2}}\right) \simeq\left(\frac{1}{\rho_{g}} \frac{1}{R_{d}} \frac{G M\left(g h^{2}\right)}{R_{d}^{3}}\right) /\left(\frac{G M}{R_{d}^{2}}\right)\\=h^{2} / R_{d}^{2}\ll1\]

因此在薄盘中, 径向压力梯度远远小于重力, 压力梯度不是主要的,
但在厚盘中需要考虑.

离心项写为:

\[\frac {v_{\varphi}^2}{R}\simeq \frac{GM}{R^2}\ \ \longrightarrow\ \  v_{\varphi}=(GM/R)^{1/2}\]

在开普勒盘中, 角速度为:

\[\Omega=v_{\varphi}/R=(GM/R^3)^{1/2}\]

考虑角动量. 吸积盘面密度为

\[\sum = \int \rho dz\]

连续方程写为 :

\[\frac{\partial \rho}{\partial t}+\nabla\cdot(\rho u)=0\]

稳态, 在柱坐标中, \(r\)方向的连续方程为 :

\[\frac 1 R \frac{\partial}{\partial R}({R\rho v_R})=0\]

因此:

\[\int \frac 1 R \frac{\partial}{\partial R}({R\rho v_R})dz=0\]

由于密度取决于z, 但是速度不取决于z, 改写为面密度:

\[\frac{d}{dR}(R\Sigma v_R)=0\]

吸积率可以表示为单位面密度成径向速度的和, 因此吸积率为:

\[\dot m=- 2\pi R\Sigma v_R\]

因此:

\[R\Sigma v_R=-\dot m/2\pi\]

计算角动量, 这时候的粘性力是重要的. 单位面积上的角动量为 :

\[J = \Sigma R^2\Omega\]

单位面积的粘滞力矩为 :

\[\frac{\partial}{\partial t}\left(\sum R^{2} \Omega\right)+\frac{1}{R} \frac{\partial}{\partial R}\left(R \cdot \sum R^{2} \Omega \cdot V_{R}\right)=G \\ \ \ \ \ \ \ \ \ \ \ \ \ \ \ \ \ \ \ \ \ \ \  \text { (viscous torgue per unit area) }\]

力矩等于\(\partial L/\partial t\); 第二项为散度算子在柱坐标中的展开.
其中角动量守恒:

\[\frac \partial {\partial t}\left(\sum R^{2} \Omega\right)=0\]

因此力矩为 :
\begin{equation}
  \begin{array}{c}
    G \cdot 2 \pi R \cdot d R=d \Gamma \quad(\Gamma: \text { viscous torgue }) \\
    G=\frac{1}{2 \pi R} \frac{d \Gamma}{d R}
  \end{array}  
\end{equation}

有速度差就有速度剪切. 因此:

\[\frac{d v_{\varphi}}{dR}=\frac d {dR}(\Omega R)=\Omega+R\frac{d\Omega}{dR}\]

其中\(\frac{d\Omega}{dR}\)就是剪切. 粘滞力可以写为:

\[\text{viscous force}=\mu R\frac{d\Omega}{dR}\]

因此粘滞力矩可以写为:
\begin{equation}
  \begin{aligned}
    \Gamma(R) 
    &=\iint\left(\mu R \frac{d \Omega}{d R}\right) \cdot(R d \varphi d z) \cdot R \\
    &=R^{3} \frac{d \Omega}{d R} \cdot 2 \pi \cdot \int \rho \nu d z \\
    &=2 \pi \nu \sum R^{3} \frac{d \Omega}{d R}
  \end{aligned}
\end{equation}

因此力矩可以写为 :

\[G(R)=\frac{1}{R} \frac{d}{d R}\left(\nu \sum R^{3} d \Omega / d R\right)\]

\[\frac{1}{R} \frac{d}{d R}\left(\sum R^{3} \Omega V_{R}\right)=\frac{1}{R} \frac{d}{d R}\left(\nu \Sigma R^{3} \frac{d \Omega}{d R}\right)\]

\[\sum R^{3} \Omega v_{R}=\nu \sum R^{3}\frac{d \Omega}{d R}+C\]

当没有剪切的时候, \(\frac{d \Omega}{d R}=0\), 确定系数C为:

\[C=\sum R_{*}^{3} \Omega_{*} v_{R_{*}}=-\dot{m} / 2 \pi \cdot R_{*}^{2} \Omega_{*}\]

代回去, 得到:

\[-\dot{m} / 2 \pi \cdot R^{2} \Omega=\nu \sum R^{3} \frac{d \Omega}{d R}-\dot{m} / 2 \pi \cdot R_{*}^{2} \Omega_{*}\]

\[\Omega=\left(G M / R^{3}\right)^{1 / 2}\]

解得:

\[\nu \Sigma=\frac{\dot m}{3 \pi}\left(1-\left(\frac{R_{*}}{R}\right)^{1 / 2}\right)\]

因此\textbf{粘滞系数正比于吸积率}. 接下来估计能量和角动量的变化率.

能量变化率为:
\begin{equation}
  \begin{aligned}
    \frac{d E}{d t} 
    &=-\int_{(\text {per unit area })}\mu(Rd\Omega/dR)^2dz 
    \\&=-\nu \sum R^{2}(d \Omega / d R)^{2} 
    \\&=-\frac{3 G M \dot{m}}{4 \pi R^{3}}\left(1-\left(\frac{R_{*}}{R}\right)^{1 / 2}\right)
    \end{aligned}
\end{equation}


角动量为:
\begin{equation}
  \begin{aligned}
    L &=\int_{R_{*}}^{\infty}-\frac{d E}{d t} 2 \pi R d R 
    \\&=G M \dot{m} / 2 R_{*} 
    \\&=\text { half of } \frac{G M}{R_{*}} \dot{m}
    \end{aligned}
\end{equation}


从上面的分析我们知道

\[L\longrightarrow \dot m \longrightarrow  \nu\]

因此吸积率越大, 粘滞力越大, 角动量越大, 盘越不稳定.

\hypertarget{3-ux8584ux76d8ux6f14ux5316}{%
\paragraph{3. 薄盘演化}\label{3-ux8584ux76d8ux6f14ux5316}}

综合上面的式子, 得到diffusion equation :

\[\frac{\partial \Sigma}{\partial t}=\frac{3}{R} \frac{\partial}{\partial R}\left(R^{1 / 2} \frac{\partial}{\partial R}\left(\nu \Sigma R^{1/2}\right)\right)\]

\[\sum \propto \exp \left(-\frac{R^{2}}{12 \nu t}\right)\]

这就是不稳定薄盘的演化方程.

\hypertarget{46-stellar-structure}{%
\section{Stellar Structure}\label{46-stellar-structure}}

compressible fluid and spherical symmetric

\[-\vec{\nabla}p+\rho \vec{g}=0\\
-\frac{dp}{dr}-\rho g = 0\]

重力g表示为:

\[g=\frac{G}{r^2}\int_{0}^{r}4\pi r'^2 \rho dr'\]

也就是说对球壳\(dr'\)积分,得到重力g.

在球坐标中表示:

\[\frac{dp}{dr}+\rho \frac{G}{r^2}\int_0^{r} 4\pi r'^2dr'=0\]

对\(r\)求导:

\[\frac{d}{dr} \left( \frac{dp}{dr}r^2\frac{1}{\rho} \right)+4\pi G\rho r^2=0\]

整理得:

\[\frac{1}{r^2} \frac{d}{dr}  \left( r^2\frac{1}{\rho}\frac{dp}{dr}   \right)=-4\pi G\rho\]

another approach

\[-\vec{\nabla}p+\rho \vec g =0 \ \ , \ \ \vec g=-\vec\nabla \Phi\]

对两边求微分,有:

\[\vec \nabla (\frac{1}{\rho}\vec\nabla p)=\vec \nabla \vec g=-\vec\nabla ^2 \Phi\]

引力势能与引力的关系:

\[\oint \vec gd\vec s=\int -\nabla^2\phi dV\]

积分得到:

\[\frac{G\rho V}{r^2}4\pi r^2=\nabla^2 \phi V\]

得到:

\[\nabla^2 \phi = 4\pi G \rho\]

polytropic gas:

多方关系:

\[p = \kappa \rho ^{\gamma}=\kappa\rho^{(1+1/n)}\]

其中n为多方指数。对于每单位体积的能量:

\[u=np\]

对于恒星对流区,以H原子为主,因此:

\[\gamma = 5/3;\ \ n=1.5\\\]

对于辐射区以光子为主:

\[\gamma=4/3;\ \ n=3\]

边界条件为:isothermal边界

\[p\propto\rho; \ \ n\rightarrow \infty\]

incompressible边界

\[p \propto \rho ,\ \ n\rightarrow 0\]

方程为:

\[K\left(1+\frac{1}{n}\right)\frac{1}{r^2}\frac{d}{dr}\left[r^2\rho^{(-1+1/n)}\frac{d\rho}{dr}\right]+4\pi G\rho=0\]

设:

\[a=\frac{4\pi G}{K(1+n)}\rho_c^{1-1/n} \\
z=ar\\
u=(\rho/\rho_c)^{1/n}\]

得到:

\[\frac{1}{z^2}\frac{d}{dz}(z^2\frac{du}{dz})+u^n=0\]

\section{Boundary Layer}

边界层是把靠近边界的地区考虑粘滞力,

\begin{equation}
  \begin{aligned}
  &\partial u_{x} / \partial x+\partial u_{y} / \partial y=0 \\
  &u_{x} \frac{\partial u_{x}}{\partial x}+u_{y} \frac{\partial u_{x}}{\partial y}=
  -\frac{1}{\rho} \frac{\partial p}{\partial x}
  +\nu\left(\frac{\partial^{2} U_{x}}{\partial x^{2}}
  +\frac{\partial^{2} u_{x}}{\partial y^{2}}\right) \\
  &u_{x} \frac{\partial u_{y}}{\partial x}+u_{y} \frac{\partial u_{y}}{\partial y}
  =-\frac{1}{\rho} \frac{\partial p}{\partial y}
  +\nu\left(\frac{\partial^{2} u_{y}}{\partial x^{2}}
  +\frac{\partial^{2} U_{y}}{\partial y^{2}}\right)
  \end{aligned}
\end{equation}

设边界层为\(\delta\), 平板长度为\(l\), 化简得到:

\[\delta\simeq \left(
\frac{\nu l}{u_x}
\right)^{1/2}\simeq Re^{-1/2}l\]

当雷诺数越大时, 边界层越小. 同时也可以知道:

\[Re^{-1/2}\ll 1\]

对\(u_y\)方向也同理. 最后化简得到的方程为 :

\[\left\{\begin{array}{l}\frac{\partial u_{x}}{\partial x}+\frac{\partial u_{y}}{\partial y}=0 \\u_{x} \frac{\partial u_{x}}{\partial x}+u_{y} \frac{\partial u_{x}}{\partial y}=-\frac{1}{\rho} \frac{\partial p}{\partial x}+\nu \frac{\partial^{2} u_{x}}{\partial y^{2}} \\0=-\frac{1}{\rho} \frac{\partial p}{\partial y}\end{array}\right.\]

Prandtl's equation



\chapter{Gas Dynamics}
\section{thermodynamics}

energy per unit volume:

\[u = np\]

pdytropic index:

\[\gamma=1+\frac{1}{n}\]

energy per unit mass

\[\varepsilon =u/p=c_VT\]

\(c_V\): specific heat per unit mass

\[p = Nk_BT\]

N: number desity, \(\rho = m_p N\)

\[p = \frac{\rho}{m_p}k_B T=\rho\frac{k_BN_A}{m_pN_A}T\]

there: \(k_BN_A=R=8.314;\ \ m_pN_A=\mu\) mass per mole

\[p = \rho \frac{R}{\mu}T\]

\[c_VT=u/\rho=np/\rho=n\frac{R}{\mu}T\]

有:

\[c_V = nR/\mu = \frac{1}{\gamma-1}\frac{R}{\mu}\]

带入压力方程:

\[p = (\gamma-1)\rho c_VT\]

the first law of thermodynamics:

\[d\varepsilon = Tds+pd(\frac{1}{\rho})\]

带入p:

\[c_VdT = Tds+(\gamma-1)c_VT\frac{1}{\rho}d\rho\\
d\ln T=\frac{1}{c_V}ds+(\gamma-1)d\ln \rho\]

得到:

\[ds = c_Vd\ln \frac{T}{\rho^{\gamma}}\\
s = c_V\ln\frac{p}{\rho^{\gamma}}+s_0\]

adiabatic: \(ds = 0\)

\section{Sound Wave}

由于

\[\frac{\partial \rho}{\partial t}+\nabla\cdot(\rho u)=0\\
\frac{\partial u}{\partial t}+u\cdot\nabla = -\frac{1}{\rho}\nabla p\]

微扰:

\[\rho=\rho_0+\rho_1 \ \ p = p_0+p_1\ \ u=0+u\]

代入线性微扰:

\[\frac{\partial\rho_1}{\partial t}+\rho_0\nabla\cdot u_1=0\\
\frac{\partial u}{\partial t}=-\frac{1}{\rho_0}\nabla p_1 
\label{eqo}\]

解得:

\[\frac{\partial ^2 \rho_1}{\partial t ^2}-\nabla^2p_1=0\]

记\(dp/d\rho=C_s^2\), \(p_1=dp/d\rho\times\rho_1\):

\[\frac{{\rm{d}}^2\rho_1}{{\rm{d}}t^2}-C_s^2\nabla^2\rho_1=0\]

代入波动解:

\[\rho_1=\hat \rho e^{i(kx-\omega t)}\]

解得:

\[-\omega^2+C_s^2k^2=0\ \ \rightarrow\ \ \omega=\pm C_s k\]

相速度:

\[\vec c_p=\frac{\omega}{k}=\pm c_s\hat k\]

群速度:

\[\vec c_g = \vec \nabla_k \omega=\pm c_s \hat k\]

将波动解代入{[} \(\ref{eqo}\) {]}:

\[-{\rm i}\omega\hat u=\frac{1}{\rho_0}{\rm i} \vec k \hat p\]

则: \(\hat u = \hat p\),两个方向平行

多方过程:

\[p=\kappa \rho^{\gamma}\]

得到:

\[c_s=(\gamma \frac p\rho)^{1/2}=(\gamma\frac{R}{\mu}T)^{1/2}\]

当\(\gamma=1\), 即为等温过程时:

\[c_s=(\frac p \rho)^{1/2}\]

\section{Shock}

\section{Blast Wave}

\section{Galatic Jet}

\section{Sphere Wind and Accretion}



\chapter{Waves}
\section{water wave}

unperturbed:

\[-\nabla p_0+\rho g = 0\]

浮力定律:

\[p_0=p_a+\rho g(h-y)\]

微扰欧拉方程:

\[\frac{\partial u_i}{\partial t}+u_i\nabla u_i = -\frac{1}{\rho}\nabla p_1\]

其中\(u_i\nabla u_i =0\) (微扰) ,有

\[\frac{\partial u_i}{\partial t} = -\frac{1}{\rho}\nabla p_1\]

取旋:

\[\frac{\partial \omega_1}{\partial t} = 0\]

速度分解:\(u_1 = v_1+v_2\),

\[\nabla \times v_1 = 0 \ \ \ \ \frac{\partial v_1}{\partial t}\neq0\\
\nabla \times v_2 = 0 \ \ \ \ \frac{\partial v_2}{\partial t} = 0 \ \ (no\ waves)\]

对于上面的公式,\(v_1\)旋度为0,有波动行为,有速度势\(\phi\)

\[v_1 = \nabla \phi\]

\[\nabla v_1 = 0\]

\[\nabla ^2\phi = 0\]

kinematic B. C. :
\(\partial\phi/\partial y = d\xi/dt=\partial\xi/\partial t+\partial\phi/\partial x\cdot\partial \xi/\partial x\)

dynamic B. C. : \(p_0+p_1=p_a\) and \(p_a = \rho g\xi\)

At \(y = h\):

\[\frac{\partial ^2 \phi}{\partial t}+g\frac{\partial \phi}{\partial y} = 0\]

波动解的形式为: \(\phi = \hat\phi \exp i(kx-\omega t)\):

解得

\[\omega ^2 = gk \tanh (kh)\]

当为深水波时:\(kh\gg1\), 有:
\begin{equation}
  \begin{array}{l}
    \omega ^2=gk\\
    c_p=\frac \omega k=\sqrt{g/k}\\
    c_g=\frac{\partial \omega}{\partial k}=\frac 1 2 \sqrt{g/k}
    \end{array}
\end{equation}


其中, \(c_p\)为相速度, \(c_g\)为群速度. 记波动速度为c,
一个波动周期为\(\gamma\), 则有: \(\omega=2\pi/\gamma=kc_p\);
一个周期内有多少波动周期称为波数, 波数为: \(k=2\pi/\lambda\) ;
波的移动速度又称为相速度,
而当不同频率的波叠加时的波包移动速度即为群速度\(c_g\) .


\section{Internal wave}

\hypertarget{1-internal-gravity-wave-g-mode}{%
\paragraph{1. Internal Gravity wave (g
mode)}\label{1-internal-gravity-wave-g-mode}}

流体密度是线性的 :

\[\rho_0=\bar \rho +\frac{d\rho_0}{dz}z\]

对于线性部分的连续方程 :

\[\frac{d\rho_0}{dt}=0;\ \ 
\nabla\cdot u = 0\]

\[\frac{\partial \rho}{\partial t}+u\cdot \nabla\rho+\rho\cdot\nabla u=0\]

欧拉方程:

\[\bar \rho\frac{du}{dt}=-\nabla p +\rho g\]

对于扰动部分展开 , 我们注意到\(u_1\cdot\nabla u_1\)为二阶微扰,
因此连续方程和欧拉方程可写为:

\[\frac{\partial \rho_1}{\partial t}+u_z\frac{d\rho_0}{dz}=0 ,
\ \ \ \ \ \ \ \ \ \ \ \nabla\cdot u=0\\
\bar \rho \frac{\partial u_1}{\partial t}=-\nabla p_1+\rho_1 g,
\ \ -\nabla p_0+\rho_0 g=0\]

联立得到方程:

\[\frac{\partial^2}{\partial t^2}\nabla^2u_z^2+N^2\nabla^2_{\perp}u_z=0\\
N^2=-\frac{g}{\bar \rho}\frac{d\rho_0}{dz}\]

其中\(\nabla_{\perp}=\frac{\partial}{\partial x}+\frac{\partial}{\partial y}\),
设波动解为:

\[u_z=\hat u_z\exp i(k\cdot x-\omega t)\]

解得:

\[\omega = \pm N\frac{k_\perp}{k}\]

其中\(k^2=k_x^2+k_y^2+k_z^2=k_{\perp}^2+k_z^2\)

群速度和相速度为:

\[c_p = \frac \omega k \hat k\\
c_g=\nabla_k \omega\\
c_g\cdot c_p=0\]

也就是说群速度垂直于相速度.

\hypertarget{2-internal-inertial-wave}{%
\paragraph{2. Internal Inertial Wave}\label{2-internal-inertial-wave}}


\section{Planetary Wave}

\hypertarget{0-shallow-water}{%
\paragraph{0. Shallow Water}\label{0-shallow-water}}

\(h<<L\) and \(\nabla u = 0\) , in other word :
\(u_z/u_x \approx h/L << 1\)

so \(u_x\) and \(u_y\) is independent of z

对于大气, 海洋: 浅水波, 有\(h \ll L\), 对于不可压流体 :

\[\nabla\cdot u = 0\]

\[\frac{u_z}{u_x}\simeq\frac h L\ll 1\]

\[h=h_0+h_1(x,y,t)\]

假设\(u_x, u_y\)与z无关, 非扰动项:

\[p_0=p_a+\rho g(h_0-z)\]

扰动项有:

\[p_1=\rho gh_1\]

\[\frac {\partial p_1}{\partial x}=\rho g\frac{\partial h_1}{\partial x}\\
\frac {\partial p_1}{\partial y}=\rho g\frac{\partial h_1}{\partial y}\\\]

因此

\[u_z = \frac{dh}{dt}=\frac{\partial h}{\partial t}+u_x\frac {\partial h}{\partial x}+u_y\frac {\partial h}{\partial y}\]

由于与z无关, 有:

\[\frac{\partial u_x}{\partial x}+\frac{\partial u_y}{\partial y}=-\frac{\partial u_z}{\partial z}\]

两边对z积分, 得到:

\[u_z = -h\nabla_{//}u\]

\[\frac {dh}{dt}-u_z=0\\
\frac{dh}{dt}+h\nabla\cdot u_{//}=0\]

展开, 得到:

\[\frac{\partial h}{\partial t}+\nabla\cdot(hu_{//})=0\]

\hypertarget{1-kelvin-wave}{%
\paragraph{1. Kelvin Wave}\label{1-kelvin-wave}}

mean flow : \(u_{x_0}=U\)

\[\left\{\begin{array}{l}\frac{\partial u_{x}}{\partial t}+U \frac{\partial u_{x}}{\partial x}=-\frac{1}{\rho} \frac{\partial p_{1}}{\partial x}=-g \frac{\partial h_{1}}{\partial x} \\\frac{\partial u_{y}}{\partial t}+U \frac{\partial u_{y}}{\partial x}=-\frac{1}{\rho} \frac{\partial p_{1}}{\partial y}=-g \frac{\partial h_{1}}{\partial y} \\
\frac{\partial h_{1}}{\partial t}+h_{0}\left( \frac{\partial u_{x}}{\partial x}+\frac{\partial u_{y}}{\partial y}\right)+U \frac{\partial h_{1}}{\partial x}=0 \end{array}\right.\]

有形式解 :

\[u_{x}=\hat{u}_{x} e^{i\left(k_{x} x+k_{y} y-\omega t\right)} \\u_{y}=\hat{u}_{y} e^{i\left(k_{x} x+k_{y} y-\omega t\right)} \\h_{1}=\hat{h} e^{i\left(k_{x} k+k_{y} y-\omega t\right)}\]

即:

\[A \left(\begin{matrix}
   \hat u_x \\
   \hat u_y \\
   \hat u_z
  \end{matrix}\right)=0\]

\[A=\left(\begin{array}{ccc}-i \omega+i U k_{x} & 0 & i g k_{x} \\0 & -i \omega+i 0 k_{x} & i g k_{y} \\i h_{0} k_{x} & i h_{0} k_{y} & -i \omega+i v k_{x}\end{array}\right)\]

当该方程组有解时, \(\det (A)=0\)

解得:

\[\omega = Uk_x\pm(gh_0(k_x^2+k_y^2))^{1/2}\]

\hypertarget{2-poincare-wave}{%
\paragraph{2. Poincare Wave}\label{2-poincare-wave}}

\[\vec \Omega = \Omega[\cos (\theta\hat e_y)+\sin(\theta\hat e _z)]\]

and
\begin{equation}
  2u\times\Omega  =\left|
 \begin{matrix}
   \hat e_x & \hat e_y & \hat e_z \\
   2u_x & 2u_y & 2u_z \\
   0 & \Omega\cos\theta & \Omega\sin\theta
  \end{matrix}
  \right|
  =\left[
 \begin{matrix}
   2\Omega\sin (\theta u_y)-2\Omega\cos(\theta u_z) \\
   -2\Omega\sin(\theta u_x) \\
   2\Omega\cos(\theta u_x)
  \end{matrix}
  \right]
\end{equation}


Neglect \(\cos \theta\) terms on high latitude region :
\begin{equation}
  \begin{aligned}
    2u\times\Omega  &=2\Omega\sin \theta_y\hat e_x-2\Omega\sin\theta u_x \hat e_y\\
    &=f\ u_y\hat e_x-f\ u_x\hat e_y
    \end{aligned}
\end{equation}

Corilis number: \(f=2\Omega\sin \theta\)

\begin{equation*}
     \begin{cases}
         \frac{\partial u_x}{\partial t} = -g\frac{\partial h_1}{\partial x}+fu_y \\
         \frac{\partial u_y}{\partial t} = -g\frac{\partial h_1}{\partial x}-fu_x  \\ 
         \frac{\partial h_1}{\partial t}+h_0(\frac{\partial u_x}{\partial x}+\frac{\partial u_y}{\partial y}) = 0
     \end{cases}
 \end{equation*}

we assume the form of the solution :

\begin{equation*}
     \begin{cases}
         u_x = \hat u_x\exp i(k_x x+k_y y-\omega t)\\
         u_y = \hat u_y\exp i(k_x x+k_y y-\omega t)\\ 
         h_1 =\ \hat h_1\exp i(k_x x+k_y y-\omega t)
     \end{cases}
 \end{equation*}

带入上述方程组 :

\[A=
\left[
 \begin{matrix}
   -i\omega & -f & igk_x \\
   f & -i\omega & igk_y \\
   ih_0k_x & ih_0k_y & -i\omega
  \end{matrix}
  \right]\]

即:

\[A \left[
 \begin{matrix}
   \hat u_x \\
   \hat u_y \\
   \hat h_1
  \end{matrix}
  \right]=0\]

由于\(\det(A)=0\),有:

\begin{equation}
\begin{array}{cr}
&\left|
 \begin{matrix}
   -i\omega & -f & igk_x \\
   f & -i\omega & igk_y \\
   ih_0k_x & ih_0k_y & -i\omega
  \end{matrix}
  \right|\\
&=-i\omega
\left|
 \begin{matrix}
   -i\omega & igk_y \\
   ih_0k_y & -i\omega
  \end{matrix}
\right| 
+f
\left|
 \begin{matrix}
   f & igk_y\\
   ih_0k_x & -i\omega 
  \end{matrix}
\right|
+
igk_x
\left|
 \begin{matrix}
   f & -i\omega\\
   ih_0k_x & ih_0k_y 
  \end{matrix}
\right| \\

& = i\omega^3-i\omega f^2-i\omega gh_0k\\
& = 0
\end{array}
\end{equation}
解得:

\[\omega^2=f^2+gh_0k\]

即:

\[\omega=\pm\sqrt{f^2+gh_0k}\]

\hypertarget{3-rossby-wave}{%
\paragraph{3. Rossby Wave}\label{3-rossby-wave}}

科氏力:

\[f=2\Omega\sin \theta\]

and

\[Rd\theta=dy\]

\(\beta\)平面近似:

\[f=f_0+\beta y\\\]

其中:

\[f_0=2\Omega\sin \theta\\
\beta = \partial f/\partial y\]

综合上以上式子,

\[\beta = 2\Omega \cos \theta/R\]

因此
\begin{equation}
  \begin{array}{l}
    \frac{\partial U_{x}}{\partial t}=-g \frac{\partial h_{1}}{\partial x}+f u_{y}: \partial / \partial y 
    \\\omega^{3}-\left(g h_{0} k^{2}+f_{0}^{2}\right) \omega-\beta g h_{0} k_{x}=0
  \end{array}
\end{equation}

对于运动缓慢的大气长波:

\[\omega\simeq-\frac{\beta k_x}{k^2+f^2_0/gh_0}\]

x方向的相速度:

\[c_{px}=\frac \omega k \frac {k_x} k<0\]

因此Rossy wave是向西方向传播的.



\chapter{Instabililty}

\section{Gravitational Instability}

微扰方程:

\begin{equation}
     \begin{cases}
        \frac{\partial \rho_1}{\partial t}+\rho_0\nabla u_1=0\\
         \rho_0\frac{\partial u_1}{\partial t}=-\nabla p_1-\rho_0\nabla \phi_1 \\ 
         \nabla^2\phi=4\pi G\rho_1
     \end{cases}
 \end{equation}

代入\(p_1=c_s^2\rho_1\)以及\(c_s^2=dp/d\rho\)


\section{Rotating Flow Instability}
\section{Convective Instabililty}



\chapter{Turbulence}
\section{Concepts from phenomenology}

动量方程:

\[\frac{\partial \vec u}{\partial t}+u\cdot\nabla u=-\frac{1}{\rho}\nabla p+\nu \nabla^2u\]

涡度方程:

\[\frac{\partial \vec \omega}{\partial t}+u\cdot\nabla \omega=\omega\cdot\nabla u+\nu \nabla^2\omega\]

其中\(\omega\cdot \nabla u\)为拉伸项。


\section{Reynolds Average and Turbulent Viscosity}

设:

\[u=\bar{u}+u'\]

动量方程:

\[\frac{\partial u_i}{\partial t}+u_j\frac{\partial u_i}{\partial x_j}=-\frac{1}{\rho}\frac{\partial p}{\partial x_i}+\nu\frac{\partial}{\partial x_j}(\frac{\partial u_i}{\partial x_j})\]

代入微扰假设:

\[\frac{\partial \bar u_i}{\partial t}+\bar u_j\frac{\partial \bar u_i}{\partial x_j}
 = 
\overline{ -u'_j\frac{\partial u'_i}{\partial x_j}}-\frac{1}{\rho}\frac{\partial \bar p}{\partial x_i}+\nu\frac{\partial}{\partial x_j}(\frac{\partial \bar u_i}{\partial x_j})\]


\section{Cascade and Scaling Law}

Largest Scale : \(l\) ,\(\tau_l\), \(u_l\)

Any Other Scale : \(\lambda\) , \(\tau_{\lambda}\) , \(u_{\lambda}\)

各个尺度下能量传输速率相等:

\[\varepsilon \sim u_l^2/\tau_l\sim u_{\lambda}^2/\tau_{\lambda}\]

由于\(\tau_l\simeq u_l/l\), 代入上式,整理得:

\[u_{\lambda}/u_l \sim (\lambda/l)^{1/3}  \\
\tau_{\lambda}/\tau_{\lambda} \sim (\lambda/l)^{2/3}\]

粘滞系数得量纲为:\([m^2s^{-1}]\) ,即\(\nu _l \sim u_l \cdot l\) :

\[\nu_\lambda /\nu_l \sim \frac{u_\lambda \lambda}{u_l l}\sim (\frac{\lambda}{l})^{4/3}\]

湍流粘滞系数\(\nu_t\)为:

\[\nu_t\sim\frac{\int_0^l \nu_\lambda d\lambda}{\int_0^ld\lambda} \sim\nu_l\]

若外力强迫是时间周期尺度,产生共振:

\[\nu_t \sim \nu_\lambda |_{\tau _{\lambda}=T}\]

由于:

\[\nu_\lambda/\nu_l\sim(\lambda/l)^{4/3}\sim(\tau_\lambda/\tau_l)^2\]

当\(\lambda\sim T\):

\[\nu_t/\nu_l \sim (T/\tau_l)^2\]

得出:

\[\nu_t \sim \nu_l(T/\tau_l)^2\]

tidal turbulence \(\nu_t<<\nu_l\),对于 smallest scale :
\(\eta , \tau_\eta, u_\eta\) :

\[\varepsilon \sim \frac{u_l^2}{\tau_l}\sim \frac{u_\eta^2}{\tau_\eta}\sim \nu\tau_\eta^{-2}\ \ and \ \ [\varepsilon]=m^2s^{-3}\]

\[\frac{u_{\eta}^2}{\tau_{\eta}}\sim\nu\tau_{\eta}^{-2}\rightarrow u_\eta\cdot \eta/ \nu\sim1\]

\[u_l^2/\tau_l \sim \nu\tau_\eta^{-2}\rightarrow\eta/l\sim(u_l\cdot l/\nu)^{-3/4}\sim {\rm{Re}}_l^{-3/4}\]

因此:

\[u_\eta/u_l \sim {\rm{Re}} _l^{-1/4}\\
\tau_\eta/\tau_l\sim{\rm{Re}}_l^{-1/2}\]

Example: 设有云朵:\(l\sim10^3 \rm m\), \(u_l\sim 1\rm m/s\),
\(\nu\sim 10^{-5} \rm m^2s^{-1}\)

\begin{equation}
  \begin{aligned}
    {\rm{Re}}_l 
    &\sim \frac {u_l\cdot l}{\nu}\sim1\times 10^3/10^{-5}=10^8 \\
    \eta 
    &\sim {\rm{Re}}_l^{-3/4}\cdot l =(10^8)^{-3/4}\times10^3 {\rm m}=10^{-3}{\rm m}\\
    u_\eta 
    &\sim {\rm{Re}}_l^{-1/4}\cdot u_l=(10^8)^{-1/4}\times1 \rm m/s = 10^{-2}\rm m/s
    \end{aligned}
\end{equation}

\section{Energy Spectrum}



\chapter{Mangeto-hydrodynamics (MHD)}

Introduction to MHD

\section{Equation}

在低速流动:\(u<<c\)

流体参考系和实验室参考系的关系:

\[E'=E+U\times B\\
B'=B\]

法拉第定律:

\[\nabla\times E=\frac{\partial B}{\partial t}\rightarrow B_0/t_0\sim E_0/l_0\rightarrow E_0/U_0\]

\[\mu\epsilon E/t_0 \ \  /\ \  B/l_0\sim (u_0/c)^2<<1\]

位移电流小,去掉

欧姆定律:

\[j'=\sigma E'=\sigma(E+U\times B)\]

\[\nabla \times B=\mu j\]

\[\partial B/\partial t=-\nabla\times E\]

\(\eta\):磁扩散系数

磁感应项+磁扩散项

\[R_m=\frac{\nabla \times (U\times B)}{\eta \nabla^2B}=
\frac{U_0 B/l_0}{\eta B/l_0^2}=u_0l_0/\eta\]

不可压缩流体:速度散度=0

\(\eta=0\) : 阿尔文磁冻结定理(\(R_m\rightarrow \infty\))

磁力线冻结-\/-\textgreater 阿尔文波 1942

\[\rho(\frac{\partial u}{\partial t}+u\cdot\nabla u)=-\nabla p\]

洛伦兹项:应力张量,正+剪切

切向投影s

法向投影n-\/-\textgreater 张力

理想磁流体:没有粘性和磁扩散,

不可压,微扰

磁力线与群速度方向一样

群速度:\(v_a\)阿尔文速度

u/v\_a=M\_a : 磁马赫数

可压缩:磁声波 \(c_g=\sqrt{c_s^2+v_a^2}\)

\end{document}
